\chapter{Introduction}

Beginning in HWS2023, we held a weekly reading group to work through the proof of the classification of simple Lie groups.
Famously, there is a complete classification: every simple Lie group belongs to one of four infinite families or is one of five exceptions.
What surprised us is that there was no single book (that we found) who set itself the task of proving this result from the beginning in full.
The purpose of this seminar report is to consolidate our understanding by putting together our sources into a single proof.
We found that quite often the theorems were written with an eye towards further developments.
This was especially true for parts of the representation theory of Lie algebras.
We felt certain techniques were avoided or others emphasised because of the role that they play e.g. in linear algebraic groups over finite fields.
So a secondary aim is to reduce the ideas to the simplest form necessary to prove the classification.
We do not intend to be fanatical; if an idea is clearer in its general form, so be it.
Finally, we are geometers and we do not try to hide our biases about what we find interesting or worth exploring.

We imagined audience of this report is ourselves one year ago.
If we could send this through time a year into the past, then it would have served as the main source for our reading group. 
We assume therefore that a reader is a graduate student familiar with manifold theory but has never formally studied Lie groups.
Basic linear algebra is also a given.

TODO: Lit review
\cite{Warner1983} We used this for Lie group theory and the bridge to Lie algebras.
\cite{Hall2015} We used this for Lie algebra theory.
\cite{Fulton2004} We used this as a supplement for Lie algebra theory, including for the classification of Dynkin diagrams.

\cite[p.~349]{Knapp1996} has a nice quote 
\begin{quotation}
The virtue of classification is that it provides a clear indication of
the scope of examples in the subject. It is rarely a sound idea to prove
a theorem by proving it case-by-case for all simple real Lie algebras.
Instead the important thing about classification is the techniques that are
involved. Techniques that are subtle enough to identify all the examples
are probably subtle enough to help in investigating all semisimple Lie
algebras simultaneously.
\end{quotation}


Citation style:
Sharpe~\cite{Sharpe1997} numbers within chapter and section, but leaves off the chapter. So our Definition~1.1.36 means Chapter 1, Definition~1.36.