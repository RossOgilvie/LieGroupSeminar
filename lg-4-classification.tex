

\section{Classification of Lie algebras}

TODO: make sections.

I would like to gather up at the start all the structure that we need. I am thinking here in particular of inner products and Weyl reflections.

The introduction of \url{https://terrytao.wordpress.com/2013/04/27/notes-on-the-classification-of-complex-lie-algebras/} is nice.

From \url{https://en.wikipedia.org/wiki/Killing_form} there seems to be a close connection between the Killing form and decompositions of Lie algebras:
\begin{itemize}
\item On a simple Lie algebra any invariant symmetric bilinear form is a scalar multiple of the Killing form.
\item The Killing form is also invariant under automorphisms of the Lie algebra
\item The Cartan criterion states that a Lie algebra is semisimple if and only if the Killing form is non-degenerate.
\item The Killing form of a nilpotent Lie algebra is identically zero.
\item Ideals that intersect only trivially are orthogonal.
\end{itemize}
This paper \url{https://math.uchicago.edu/~may/REU2012/REUPapers/Bosshardt.pdf} from Uchicago undergrad summer projects 2012 seems to go hard on the Killing form.

If we are focused on simple Lie algebras, then can we avoid the other types?

I don't want to build representation theory.
I think we only really need the adjoint representation and $\frsl(2,\bbC)$-representations.
Maybe we can understand them as a special type of subspace.


\subsection{Cartan subalgebra}
A point of difficulty seems to be Cartan subalgebras.
In fact there are several, somewhat incompatible definitions in the literature (in different cases) to suit different circumstances.
Hall takes the simplest definition: a maximal commuting subalgebra of diagonalizable elements.
This immediately gives you a root space decomposition, commuting implies simultaneously diagonalizable.
The existence is rather easy too, but only because he begins with a real compact form for the complex semisimple algebra.
FH takes the same definition, but in Appendix D proves existence through regular elements.

Knapp takes a more principled approach, asking what we should require of a subalgebra to get a decomposition like we see in the standard examples, p83.
It takes the more general starting point of a nilpotent subalgebra h.
Not all of them lead to a really nice decomposition, but somehow maximal ones do, and these we call Cartan subalgebras.
The drawback is that we don't have diagonalizability at the start, so we have to work with generalised eigenspaces, see Prop 2.5 p88.
Here too the existence relies on regular elements Theorem 2.9.

Knapp Chapter 4 comes back to Hall's point: Cor 4.26 a compact Lie group has negative definite Killing form. Prop 4.27, gives an argument for the converse!

What drew me to Knapp though was the discussion of real simple Lie algebras.
In particular, how complex Lie algebras can arise from real ones.
The two extremes are the split real form and the compact real form.
These are constructed explicitly Cor 6.10 and Thm 6.11, but they require the root decomposition.
Other real forms (and these ones too) are described by a Cartan involution. 
It is a vector space sum on which the Killing form is resp pos and neg definite (6.26) and the brackets are behaved (6.24).
Perhaps this could be the basis of a direct proof of a compact real form.
After some googling, this was suggested by Cartan 1929 to simplify the proof, and there's a paper by Richardson (?) that carries it out.
But it's not as simple as it sounds.

Another thing to consider would be to only classify compact lie groups.
This has the advantage that then the compact real form is a natural part of the set-up.
You get basically the same list at the end.
It has most of the features you want to consider.
And then maybe you can carry the classification all the way back to the level of real Lie groups, without so much of the messy involution stuff at the end.
Go deeper on compact Lie groups; apparently their universal covers are compact, you can understand the quotient theory well.
It does feel like a limitation, when we are so close to getting all complex semisimple Lie algebras.


\begin{webonly}
\includedesmosThreeD{e275919fe1}
\end{webonly}