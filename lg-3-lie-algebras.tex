\chapter{Lie algebra}

TODO: maybe the angle here should be that Lie groups are homogeneous manifolds, so it makes sense to focus on neighbourhoods of the identity.
Maybe the result that a neighbourhood of the identity generates a connected Lie group \cite[3.18]{Warner1983}.
Then talk about the exponential map which is a local diffeomorphism from $\frg$ to $G$ at $e$.
This frames the section: we have the internal generation of a Lie group, how do we externally generate it from the tangent space?

\section{Lie Bracket}
Lie bracket coming from the adjoint action.
There are lots of names for conjugation, eg the power notation, Sharpe uses $\mathbf{Ad}$, Warner $a$. I think a two letter operator, eg $\operatorname{Cn(g)}$ would be best. Wikipedia uses $\Psi$. There's also the question of $\ad(x)$ or $\ad_x$.
I think this one shows most clearly how the bracket is encodes some infinitesimal information of the group operation.

This is Lie bracket of left-invariant vector fields.


\section{Examples}
Matrix vs abstract Lie algebras
Bracket of matrix Lie algebras is commutator

Lie algebras of all the classical groups.


\section{Correspondences}

Stuff like Lie group homomorphisms inducing Lie algebra homomorphisms.
ideals and subalgebras

I found this pdf \url{https://www.cis.upenn.edu/~cis6100/cis610-15-sl17.pdf} that connects metrics on the Lie group with inner products on the Lie algebra.


\section{Ado's theorem}
\url{https://terrytao.wordpress.com/2011/05/10/ados-theorem/}
This seems elementary, and introduces all the important classes of Lie algebras: sovlable, nilpotent, etc. But it defines complex Lie algebras at the outset? Maybe we already need to consider real vs complex Lie algebras, complexifications, etc.

Maybe we only need a weaker version, that all simple Lie algebras are matrix?
Indeed, the adjoint representation works for centerless Lie algebras.
